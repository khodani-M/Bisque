\documentclass[english]{article}

\usepackage{graphicx}
\usepackage{grffile}
\usepackage{babel}

\author{
	Mufamadi, Khodani\\
	\texttt{14197520}
	\and
	Burgers, Heinrich\\
	\texttt{150595538}
	\and
	Cheriyan, Midhun\\
	\texttt{17308632}
	\and
	Cilliers, Joshua\\
	\texttt{14267196}
	\and
	Leshaba, Harris\\
	\texttt{15312144}
	\and
	Van Hattum, Jason\\
	\texttt{15027458}
	\and
	Rambani, Unarine\\
	\texttt{14004489}
}

\title{Cos301 : Software Requirements Specification\\
	for the NavUP System\\
	}
\date{\today}
\graphicspath{{Pictures/}}

\begin{document}
	\maketitle
	\begin{figure}[!t]
		\includegraphics{up_logo.png}
	\end{figure}
	\pagenumbering{gobble}
	\newpage

	\tableofcontents
	\newpage

	\pagenumbering{arabic}
	

	\section{Introduction}
		\paragraph\indent
			

		\subsection{Purpose}
			\paragraph\indent
			The purpose of this document is to give a detailed description of the requirements for the NavUP System, it  will illustrate the purpose and complete declaration for the development of the system. It will also explain system constraints, interfaces and interactions with other external applications. This document is primarily intended to be proposed to a client for its approval 					and as well as the stakeholders and interested parties in such a system.This document  will also serve as a reference for developing the first version of the system for the development team.

		\subsection{Scope}
			\paragraph\indent
			The “NavUP System” is a  WiFi-based mobile navigation application which helps users navigate on campus to various venues to attend meetings , classes or shops, restaurants and ablution facilities based on the user’s current position and other specifications like restrictions specified by the user such as accessibility for people with disabilities or avoiding pedestrian traffic 					congestion. The application should be free to download from either a mobile phone application store or similar services.\\
			Furthermore, the software needs the campus wi-fi  or Internet and GPS connection to fetch and display results to users on their smart devices. The current location of the user should be determined both outdoors and indoors ,Various activities that uses location and movement of users can also be integrated.

		\subsection{Definition, Acronyms, and Abbreviations}
			\paragraph\indent
				This section of the SRS contains definitions, acronyms and abbreviations for the terminology used to describe our system throughout this document.
				\begin{tabular}{ |p{3cm}|p{9cm}|  }
				\hline
				\textbf{Term} & \textbf{Definition}\\
				\hline
				User & Someone who interacts with the mobile phone application \\
				\hline
				Administrator & System administrator who is given specific permission for managing and controlling the system\\
				\hline
				\end{tabular}

		\subsection{References}
			\paragraph\indent
			[1] IEEE Software Engineering Standards Committee, "IEEE Std 830-1998, IEEE Recommended Practice for Software Requirements Specifications", October 20,1998

		\subsection{Overview}
			\paragraph\indent
				\begin{tabular}{ |p{3cm}||p{11cm}|  }
				\hline
				\multicolumn{2}{|c|}{This SRS document describes the NavUp system. It is divided into three major 							sections} \\
				\hline
				Section 1 & This section describes the document purpose and the project scope.This section also includes 						the definitions, abbreviations, and references used throughout this document \\
				\hline
				Section 2 & This section provides the product Persperctive and the product functions,And also describes 						the uses of NavUp and the user characteristics.The Constraints, assumptions and dependencies for NavUp					are described in this section\\
				\hline
				Section 3 & This section describes the external interface requirements for the NavUp system and a list of functional requirements and performance requirements .Design constraints and software system attributes and as well as other requirements for the NavUp system are included in this section.\\
				\hline
				\end{tabular}

	\section{Overall Description}
		\paragraph\indent
			...
		
		\subsection{Product Perspective}
			\paragraph\indent
			
				\subsubsection{System Interfaces}
					\paragraph\indent
						...

				\subsubsection{User Interfaces}
					\paragraph\indent
						...

				\subsubsection{Hardware Interfaces}
					\paragraph\indent
						\begin{itemize}
						    \item An android phone or tablet.
						    \item An iPhone or iPad.
						\end{itemize}

				\subsubsection{Software Interfaces}
					\paragraph\indent
						The mobile application must communicate with GPS application in order to obtain the geographical location of the user when the user is not inside a building, however if the user is inside a building – the mobile application must communicate with the router software to perform triangulation to find the exact location of the user inside the building.
						The mobile application must communicate with the database software to lookup events and  activies of interest to the user and provide information related to pedestrian traffic on campus.

				\subsubsection{Communications Interfaces}
					\begin{itemize}
					    \item Cellular networks
					    \item Global Positioning Satellites (GPS)
					    \item Wireless networking (WiFi)
					    \item E-mail
					\end{itemize}

				\subsubsection{Memory}
					\paragraph\indent
					The system must be memory efficient due to the fact that mobile device do not have large amounts of memory and the application must be able run simultaneously with other applications.  
Typical android and iOS devices have 1GB of RAM or more, therefore to prevent overloading the operating system the application must use at most 50MB of RAM while the application is runnuing.
The application must use 50MB of hard disk drive space. 
			

				\subsubsection{Operations}
					\paragraph\indent
						...

				\subsubsection{Site Application Requirements}
					\paragraph\indent
						...


		\subsection{Product Functions}
			\paragraph\indent
Users will be able to use their handheld devices to navigate around the campus if they having trouble to find venues. The system will assist the user to avoid pedestrian traffic and find the quickest route to their desired venue. The system will be able to locate the user’s current location either inside or outside the building by the use of the Wi-Fi access points. The system will store different types of information such as social events and activities, venues using multiple types of devices and services.The different types of information will be store on database. 

		\subsection{User Characteristics}
			\paragraph\indent
				NavUP should have three user groups: A student or staff member, a administrator, and a guest user.
				\subparagraph{Students and staff members}
				\begin{itemize}
					\item Students and staff members are registered, and have student numbers.
					\item Students are likely to be young (Below the age of 30)
					\item Students and staff members are likely to have a high level of education.
					\item Students and staff members should have a relatively high level of technical experience, and therefore be able to use and navigate a relatively complex app.
				\end{itemize}
				\subparagraph{Guest Users}
				\begin{itemize}
					\item Guests are unregistered.
					\item The technical level and education of a guest is unknown. It might be difficult for them to navigate a complicated interface.

				\end{itemize}
				\subparagraph{Administrators}
				\begin{itemize}
					\item Administrators should have a high level of technical expertise.
					\item Administrators likely have some form of identification, such as a student number.
					
				\end{itemize}
		\subsection{Constraints}
			\paragraph\indent
			    \begin{itemize}
			        \item \textbf{Cost} - We do not have the funds to pay for expensive libraries and tools.
			        \item \textbf{Time} - Most of us are third year and honours students, and so we do not have much time to work on the project. Additionally, we only have one semester to do this.
			        \item \textbf{Skills} - Our skills are varied, but mostly undeveloped, which limits the technical complexity of our solution.
			        \item \textbf{Scope} - Our scope is defined as a navigation system for the University of Pretoria, and so our solution should be limited as such.
			    \end{itemize}

		\subsection{Assumptions and Dependencies}
			\paragraph\indent
				\begin{large}$2.5.1 $ \textbf{Assumptions}\end{large}
	\begin{itemize}
		\item The application is free for all users.
		\item Users will access the application through 				  hand-held devices.
		\item The system will be available 24/7
		\item The system will be simple.
		\item The system can locate a user inside and     				  outside the building.
		\item The system can detect pedestrian traffic  				  around the campus.
		\item User friendly system.
		\item Users are technically competent.
		\item Access to Wi-Fi.
		\item Users operating system is either Android or  			 IOS.
		\item Users will be willing to provide personal  				  information.

\end{itemize}


\begin{large}
	$2.5.2 $ \textbf{Dependencies}
	\end{large}
	\begin{itemize}
		\item Teams time and abilities. 
		\item Feedback from stakeholders. 
		\item A database to store user’s history, location 			  and social activities
		\item The application needs access to Wi-Fi.
		\item Adequate advertisement. 

	\end{itemize}



	\section{Specific Requirements}
		\paragraph\indent
			...

				\subsection{External Interface Requirements}
					\paragraph\indent
					    Since the prototype will most likely only be developed for Android systems and later expanded to iOS and other systems, this part of the specification will primarily assume the Android OS is the operating system for the system.
					    \subparagraph{System Interfaces} 
					     - The app will require system privileges to make use of certain features and information provided by the Android OS.
					    \begin{enumerate}
					        \item  \textbf{Communication Protocols}\\
					        The app will be dependant on making use of the connection to the internet provided through the device's connection to Wi-fi or mobile data. This will be necessary for the main functions such as mapping, location tracking, navigation and heatmaps (for traffic avoidance) to work.
					        \item \textbf{GPS}\\
					        We will be dependant on Android's GPS and location services to be able to accurately track both where the users are and where they are going. Access to this feature forms the backbone of the app as nearly all our subsystems are dependant on it.
					    \end{enumerate}

					    \subparagraph{User Interfaces}
					     - There will be a unified user interface which can be broken down into 5 subsystems all of which will be displayed simultaneously. The 6th subsystem (Suggestions) will tie into the navigational subsystem.
					    \begin{enumerate}
					        \item \textbf{Navigation}\\
					        Users will use the navigation subsystem to select a new destination, as well as deciding on their starting point. This information will be sent through to the mapping and location subsystems. It will eventually receive output from the mapping subsystem in the form of a route to follow which it will forward on to the display subsystem to display on screen for users.
					        \item \textbf{Location}\\
					        Users will passively interact with the location subsystem as it will autonomously track the device's location for input which the Navigation, Mapping, and Heatmap subsystems will use. It will continuously update the device's location and provide it as output to the mapping and/or navigational subsystem depending on context.
					        \item \textbf{Mapping}\\
					        This will calculate the route which the user has to take. It will accept input from the Location, Heatmaps, and the Navigation subsystems and then use that information to calculate the best possible route which it will provide as output.
					        \item \textbf{Heatmaps}\\
					        This will make use of packet sniffing or a similar technique to discover how many people are connected to routers and are gathered in certain areas at certain times. It will compare this to it's own averages gathered over time to calculate whether or not an alternative route or warning should be displayed due to high amounts of foot traffic in certain parts of campus
					        \item \textbf{Games and Events}\\
					        These events and games will be managed by users with creative privileges on the app. It's suggested that a form of validation and moderation is placed either on the people given creative privileges or on the events and games that they wish to add to the app. These games and events will then be added into the ma
