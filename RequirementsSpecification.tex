\documentclass[english]{article}

\usepackage{graphicx}
\usepackage{grffile}
\usepackage{babel}

\author{
	Mufamadi, Khodani\\
	\texttt{14197520}
	\and
	Burgers, Heinrich\\
	\texttt{150595538}
	\and
	Cheriyan, Midhun\\
	\texttt{17308632}
	\and
	Cilliers, Joshua\\
	\texttt{14267196}
	\and
	Leshaba, Harris\\
	\texttt{15312144}
	\and
	Van Hattum, Jason\\
	\texttt{15027458}
	\and
	Rambani, Unarine\\
	\texttt{14004489}
}

\title{Cos301 : Software Requirements Specification\\
	for the NavUP System\\
	}
\date{\today}
\graphicspath{{Pictures/}}

\begin{document}
	\maketitle
	\begin{figure}[!t]
		\includegraphics{up_logo.png}
	\end{figure}
	\pagenumbering{gobble}
	\newpage

	\tableofcontents
	\newpage

	\pagenumbering{arabic}
	

	\section{Introduction}
		\paragraph\indent
			

		\subsection{Purpose}
			\paragraph\indent
			The purpose of this document is to give a detailed description of the requirements for the NavUP System, it  will illustrate the purpose and complete declaration for the development of the system. It will also explain system constraints, interfaces and interactions with other external applications. This document is primarily intended to be proposed to a client for its approval 					and as well as the stakeholders and interested parties in such a system.This document  will also serve as a reference for developing the first version of the system for the development team.

		\subsection{Scope}
			\paragraph\indent
			The “NavUP System” is a  WiFi-based mobile navigation application which helps users navigate on campus to various venues to attend meetings , classes or shops, restaurants and ablution facilities based on the user’s current position and other specifications like restrictions specified by the user such as accessibility for people with disabilities or avoiding pedestrian traffic 					congestion. The application should be free to download from either a mobile phone application store or similar services.\\
			Furthermore, the software needs the campus wi-fi  or Internet and GPS connection to fetch and display results to users on their smart devices. The current location of the user should be determined both outdoors and indoors ,Various activities that uses location and movement of users can also be integrated.

		\subsection{Definition, Acronyms, and Abbreviations}
			\paragraph\indent
				This section of the SRS contains definitions, acronyms and abbreviations for the terminology used to describe our system throughout this document.
				\begin{tabular}{ |p{3cm}|p{9cm}|  }
				\hline
				\textbf{Term} & \textbf{Definition}\\
				\hline
				User & Someone who interacts with the mobile phone application \\
				\hline
				Administrator & System administrator who is given specific permission for managing and controlling the system\\
				\hline
				\end{tabular}

		\subsection{References}
			\paragraph\indent
			[1] IEEE Software Engineering Standards Committee, "IEEE Std 830-1998, IEEE Recommended Practice for Software Requirements Specifications", October 20,1998

		\subsection{Overview}
			\paragraph\indent
				\begin{tabular}{ |p{3cm}||p{11cm}|  }
				\hline
				\multicolumn{2}{|c|}{This SRS document describes the NavUp system. It is divided into three major 							sections} \\
				\hline
				Section 1 & This section describes the document purpose and the project scope.This section also includes 						the definitions, abbreviations, and references used throughout this document \\
				\hline
				Section 2 & This section provides the product Persperctive and the product functions,And also describes 						the uses of NavUp and the user characteristics.The Constraints, assumptions and dependencies for NavUp					are described in this section\\
				\hline
				Section 3 & This section describes the external interface requirements for the NavUp system and a list of functional requirements and performance requirements .Design constraints and software system attributes and as well as other requirements for the NavUp system are included in this section.\\
				\hline
				\end{tabular}

	\section{Overall Description}
		\paragraph\indent
			...
		
		\subsection{Product Perspective}
			\paragraph\indent
				...
			
				\subsubsection{System Interfaces}
					\paragraph\indent
						...

				\subsubsection{User Interfaces}
					\paragraph\indent
						...

				\subsubsection{Hardware Interfaces}
					\paragraph\indent
						...

				\subsubsection{Software Interfaces}
					\paragraph\indent
						...

				\subsubsection{Communications Interfaces}
					\paragraph\indent
						...

				\subsubsection{Memory}
					\paragraph\indent
						...

				\subsubsection{Operations}
					\paragraph\indent
						...

				\subsubsection{Site Application Requirements}
					\paragraph\indent
						...


		\subsection{Product Functions}
			\paragraph\indent
				...

		\subsection{User Charateristics}
			\paragraph\indent
				...

		\subsection{Constraints}
			\paragraph\indent
				...

		\subsection{Assumptions and Dependencies}
			\paragraph\indent
				...


	\section{Specific Requirements}
		\paragraph\indent
			...

				\subsection{External Interface Requirements}
					\paragraph\indent
						...

				\subsection{Functional Requirements}
					\paragraph\indent
						...

				\subsection{Preformance Requirements}
					\paragraph\indent
						...

				\subsection{Design Constraints}
					\paragraph\indent
						...
	
				\subsection{Software System attributes}
					\paragraph\indent
						...

				\subsection{Other Requirements}
					\paragraph\indent
						...

		
\end{document}
